\documentclass[letter, 10pt]{article}
\usepackage[utf8]{inputenc}
\usepackage[spanish]{babel}
\usepackage{amsfonts}
\usepackage{amsmath}
\usepackage[dvips]{graphicx}
\usepackage{url}
\usepackage[top=3cm,bottom=3cm,left=3.5cm,right=3.5cm,footskip=1.5cm,headheight=1.5cm,headsep=.5cm,textheight=3cm]{geometry}
%\usepackage[]{algorithm2e}

\begin{document}
\title{
    Inteligencia Artificial Avanzada\\ 
    \begin{Large}
      Ant System - Timetabling Problem
    \end{Large}
}
\author{Hernán Vargas \\ 201073009-3}
\date{\today}
\maketitle

\begin{abstract}
    %Resumen del informe en no más de 10 líneas.
\end{abstract}

\section{Introducción}\label{sec:intro}
    %Una explicación breve del contenido del informe. Es decir, detalla:
    %Propósito, Estructura del Documento, Descripción (muy breve) del Problema y
    %Motivación.

\section{Definición del Problema}\label{sec:def}
    %Explicación del problema que se va a estudiar, en que consiste, cuales son
    %sus variables, restricciones y objetivos de manera general. Variantes más
    %conocidas que existen.
    
\section{Estado del Arte}\label{sec:art}
    %Lo más importante que se ha hecho hasta ahora con relación al problema.
    %Debería responder preguntas como las siguientes ?`cuando surge?, ?`qué
    %métodos se han usado para resolverlo?, ?`cuales son los mejores algoritmos
    %que se han creado hasta la fecha?, ?`qué representaciones han tenido los
    %mejores resultados?, ?`cuál es la tendencia actual?, tipos de movimientos,
    %heurísticas, métodos completos, tendencias, etc... Puede incluir gráficos
    %comparativos, o explicativos.\\ La información que describen en este punto
    %se basa en los estudios realizados con antelación respecto al tema. Dichos
    %estudios se citan de manera que quien lea su estudio pueda también acceder
    %a las referencias que usted revisó. Las citas se realizan mediante el
    %comando \verb+\cite{ }+.  Por ejemplo, para hacer referencia al artículo de
    %algoritmos híbridos para problemas de satisfacción de restricciones
    %~\cite{Prosser93Hybrid}.

\section{Modelo Matemático}\label{sec:mod}
    %Uno o más modelos matemáticos para el problema.

\section{Representación}\label{sec:repr}
    %Representación matemática y estructura de datos que se usa (arreglos,
    %matrices, etc.), por qué se usa, la relación entre la representación
    %matemática y la estructura.

\section{Descripción del algoritmo}\label{sec:alg}
    %Cómo fue implementando, interesa la implementación más que el algoritmo
    %genérico, es decir, si se tiene que implementar SA, lo que se espera es que
    %se explique en pseudo código la estructura general y en párrafo explicativo
    %cada parte como fue implementada para su caso particular, si se utilizan
    %operadores se debe explicar por que se utilizó ese operador, si fuera el
    %caso de una técnica completa, si se utiliza recursión o no, etc. En este
    %punto no se espera que se incluya código, eso va aparte.

\section{Experimentos}\label{sec:exp}
    %Se necesita saber como experimentaron, como definieron parámetros, como los
    %fueron modificando, cuales problemas se trataron, instancias, por que
    %ocuparon esos problemas.

\section{Resultados}\label{sec:res}
    %Que fue lo que se logró con la experimentación, incluir tablas y
    %parámetros, gráficos si fuera posible, lo más explicativo posible.

\section{Conclusiones}\label{sec:conc}
    %De acuerdo a la introducción que se hizo, entregar afirmaciones RELEVANTES
    %basadas en los experimentos y sus resultados.

\section{Bibliografía}\label{sec:bib}
    %Indicando toda la información necesaria de acuerdo al tipo de documento
    %revisado. Las referencias deben ser citadas en el documento.

\bibliographystyle{plain}
\bibliography{Referencias}
\end{document} 
