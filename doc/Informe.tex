\documentclass[letter, 10pt]{article}
\usepackage[utf8]{inputenc}
\usepackage[spanish]{babel}
\usepackage{amsfonts}
\usepackage{amsmath}
\usepackage{url}
\usepackage[top=3cm,bottom=3cm,left=3.5cm,right=3.5cm,footskip=1.5cm,headheight=1.5cm,headsep=.5cm,textheight=3cm]{geometry}
\usepackage{algorithm}
\usepackage{subcaption}
\usepackage{listings}
\usepackage[noend]{algpseudocode}
\usepackage{tikz}
\usetikzlibrary{babel,arrows.meta,shapes,arrows}
\lstset{basicstyle=\ttfamily\small, numbers = left}

\renewcommand{\tt}[1]{\texttt{#1}}
\renewcommand{\it}[1]{\textit{#1}}
\renewcommand{\bf}[1]{\textbf{#1}}
\newcommand{\ra}{$\rightarrow\,\,$}
\newcommand{\la}{$\leftarrow\,\,$}

\floatname{algorithm}{Algoritmo}
\renewcommand{\algorithmicrequire}{\textbf{Input:}}
\renewcommand{\algorithmicensure}{\textbf{Output:}}

\begin{document}
\title{
  Inteligencia Artificial Avanzada\\ 
  \begin{Large}
    Ant System - Course Timetabling Problem
  \end{Large}
}
\author{Hernán Vargas \\ 201073009-3}
\date{\today}
\maketitle

\begin{abstract}
  En este trabajo se considera una simplificación del típico ``course
  timetabling problem'' atacando solo la factibilidad de la solución por medio
  de un algoritmo de hormigas.
  Se explica la generación del grafo a recorrer junto a cómo se comportarán las
  hormigas en la búsqueda de soluciones y se presentan los resultados de la
  aplicación de dicho algoritmo a un set de instancias reales selectas de la
  competencia internacional de resolución del problema.
\end{abstract}

\section{Introducción}\label{sec:intro}
En este informe se planteará e implementará un algoritmo de hormigas para la 
resolución de problemas básicos del \emph{Course Timetabling Problem} (CTP).
El problema radica en encontrar una asignación de recursos factibles para
generar un horario de clases de una escuela o universidad. Debido a su
generalidad pueden existir diversas condiciones y recursos a satisfacer
dependiendo de la instancia a resolver (por ejemplo, clases en laboratorios
específicos, número de estudiantes por sala, etc.) y además se pueden generar
un set de restricciones segundarías para discriminar el ``valor'' de una
solución (por ejemplo, penalizaciones por tener clases la última hora de un día,
tener ``ventanas'' de más de 2 horas, etc).

El propósito de esta investigación radica en verificar el funcionamiento y
rendimiento de algoritmos basados en hormigas para problemas reales y cuales son
las ventajas y desventajas de elegir esta metaheurística por sobre otras 
similares.

Este documento se organiza como sigue. En la sección~\ref{sec:def} se define el
problema junto al conjunto típico de variables y restricciones que lo conforman,
además del vocabulario básico necesario para entender correctamente este
informe. 
En la sección~\ref{sec:art} se hace un seguimiento bibliográfico del problema
con especial énfasis en los trabajos previos de resolución por algoritmos
basados en metaheurísticas y sus resultados.
La sección~\ref{sec:mod} presenta el modelo utilizado para la creación del
algoritmo y como se relacionan los parámetros definidos en éste al
funcionamiento y rendimiento del programa generado.
En la sección~\ref{sec:repr} se muestra la forma en la cual se representan
los datos computacionalmente mientras que en la sección~\ref{sec:alg} se
describe el algoritmo en sí.
Los resultados son presentados en la sección~\ref{sec:res} y las conclusiones
obtenidas de ellos se enuncian en la sección~\ref{sec:conc}.

\section{Definición del Problema}\label{sec:def}
El CTP es un \emph{scheduling problem}, es decir un problema de asignación de
recursos.
En particular en los \emph{timetabling problems} el recurso principal a asignar
es el tiempo. 
Existen variados tipos de \emph{timetabling problems}, por ejemplo el
\emph{train timetabling problem}\cite{caprara2002modeling,khan2010stochastic}
(asignar tiempos y vías para los trenes), el \emph{employee timetabling
problem}\cite{detienne2009cut,cowling2000hyperheuristic} (Asignar personal para 
ciertas tareas), el \emph{examination timetabling problem}\cite{di2000tabu}
(Asignar el tiempo en que se harán exámenes), entre otros.
En particular en CTP se busca asignar cierta cantidad de clases en un tiempo
acotado y utilizando los recursos necesarios para ello.

Variantes del CTP son tanto  el \emph{University Course Timetabling  Problem}
como \emph{School Course Timetabling Problem}. Ambos son básicamente el
mismo problema de asignación, pero se diferencian en las restricciones para
determinar la factibilidad de una solución, pues, en general, para las escuelas
se debe mantener a los estudiantes en clases por grandes periodos consecutivos
mientras que las universidades pueden ser más flexibles con sus horarios.

El CTP es un problema que se presenta típicamente en universidades y escuelas
de todo el mundo. Básicamente considera la existencia de cierto número de
eventos (clases) a realizar en un horario determinado de manera que todos sus
recursos estén disponibles en ese momento.
Una solución se considera factible si se logra que cada recurso este asignado
a un y solo un evento a la vez. La asignación de recursos en sí es un
subproblema que puede tener variadas restricciones adicionales, por lo que
generalmente el CTP se considera un problema NP-Hard.

Los recursos y restricciones a resolver en este trabajo son una versión
simplificada de aquellos presentados en \emph{The third international
timetabling competition}\cite{post2016third}. Particularmente se utilizarán
los siguientes recursos:
\begin{itemize}
  \item \bf{Tiempo:}
    Recurso principal a asignar. Todo otro recurso solo puede ocurrir en un
    tiempo a la vez.
  \item \bf{Profesor:}
    Cada evento tiene un profesor asignado.
  \item \bf{Clase:}
    Cada evento tiene una clase asignada.
  \item \bf{Sala:}
    Los eventos pueden tener una sala asignada o utilizar alguna disponible. En
    algunas instancias este recurso no existe.
\end{itemize}

Un \bf{evento} es un set de los recursos mencionados anteriormente de
forma que tanto el profesor, como la sala y la clase tengan libre el tiempo
asignado.

La restricción a evaluar será la de factibilidad, es decir que todo evento logre
una asignación correcta.

Para mantener la simplicidad del algoritmo en este trabajo se excluyen de la
evaluación instancias con otros recursos como los estudiantes (cada clase puede
tener cierto número de ellos) y restricciones opcionales para determinar la
mejor de las soluciones factibles (por ejemplo, tener clases seguidas, de
preferencia no utilizar ciertos horarios, etc).

\section{Estado del Arte}\label{sec:art}
El CTP es un problema clásico de asignación de recursos y debido a su
generalidad han existido diferentes métodos de encarar este problema.
Al ser un problema combinatorio complejo (NP-Hard) no existen algoritmos
completos que encuentren soluciones factibles en poco tiempo y por ello ha
florecido una amplia gama de técnicas incompletas de búsqueda de soluciones.

Si bien no se pueden determinar los orígenes del problema, uno de los primeros
trabajos relacionados puede ser rastreado hasta el 1963 por
Gotlieb\cite{gotlieb1963construction}, la investigación posterior es variada y
existen trabajos analizándola como \cite{schaerf1999survey}.

En los últimos años, entre las metaheurísticas comúnmente utilizadas para
encarar el problema, se destacan los \emph{Evolutionary Algorithms} (EA),
\emph{Iterated local search} (ILS), \emph{Simulted Annealing} (SA),
\emph{Tabu Search} (TS) y \emph{Ant colony Optimization} (ACO).

Los algoritmos evolutivos se basan en la mecánica de la evolución natural: la
supervivencia del más apto. En EA una solución buena tiene muchas posibilidades
de ser los padres de la siguiente generación; cada solución puede mutar y
reproducirse y así, con el paso del tiempo, mejorar las soluciones.

Podemos encontrar variadas implementaciones de EA para solucionar el CTP. 
En \cite{erben1995genetic} se resuelve un CTP simplificado como el que se
presenta en este documento. Para la representación de los cromosomas se utiliza
una n-tupla de los recursos, el procesos de mutación se basa en liberar algún
recurso y cambiarlo por otro disponible mientras que en el cruzamiento se
intercambian los recursos entre dos soluciones. Los resultados de este trabajo
enfatizan lo difícil que es encontrar los parámetros adecuados y lo costoso que
es el proceso computacionalmente. Atacando éste último problema existen trabajos
como \cite{abramson1991parallel} que aprovechan las ventajas de la computación
paralela para mejorar los resultados.

Por otra parte se destaca el trabajo de \cite{weare1995hybrid} que utilizando un
\emph{hybrid genetic algorithm} enfrenta el problema más de forma más realista: 
con restricciones segundarías variables. La factibilidad de la solución siempre
se garantiza y además se agrega al capacidad definir otras restricciones que
mejoran la aplicabilidad de la solución pero a la vez hacen al problema más
difícil.

La ILS se basa en la idea de mejorar la búsqueda local generando nuevas
soluciones de perturbaciones de la solución actual, generalmente este proceso da
mejores resultados que comenzar con soluciones aleatorias. Una descripción
detallada de un ILS puede ser encontrada en \cite{stutzle1998local}.

En \cite{rossi2002local} se aplica ILS para solucionar un CTP simplificado
basado en el problema real presentado por la universidad de Napier. El
vecindario se define como una sucesión de eventos ordenados en el tiempo y se
generan movimientos como \tt{swap} y permutaciones.

Se puede considerar a SA como una variante de la búsqueda local inspirada en el
comportamiento físico del calor. Muchos trabajos utilizan SA para atacar el CTP,
en \cite{thompson1996variants} se hace una revisión de ellos. Generalmente se
hacen optimizaciones en diferentes partes del proceso y existen versiones
secuenciales y paralelas del mismo.

El funcionamiento de TS se basa en el uso de una memoria selectiva para obtener
un buen balance entre la exploración y explotación de las soluciones. El
desplazamiento entre soluciones locales se da por el uso de movimientos y solo
cuando se mejora la solución, utilizando una lista tabú para evitar caer en
ciclos.

Al igual que SA goza de gran popularidad a la hora de enfrentar
diferentes \emph{timetabling problems}. Particularmente para CTP el trabajo de
\cite{lu2010adaptive} presenta un un TS adaptativo y en \cite{aladag2009effect}
se discute como afectan los movimientos a la generación del vecindario de
soluciones y se proponen nuevos movimientos que pueden dar mejores resultados.

Los algoritmos de hormigas y \emph{ant systems} (AS) se basan en imitar el
comportamiento de las hormigas al determinar los caminos óptimos para encontrar
comida por medio de las feromonas que dejan al desplazarse.
Han sido utilizados para resolver diversos problemas, entre ellos el CTP.

En el trabajo de Socha \emph{et al.}\cite{socha2003ant} se hace
una revisión del desempeño de estos algoritmos en el marco de la
\emph{Metaheuristics Network}\footnote{http://www.metaheuristics.org/} y en
\cite{socha2002max} de los mismos autores se presenta un \emph{MAX-MIN} AS para
solucionar el CTP. De este trabajo se destaca la forma de representar el camino
por el cual se desplazarán las hormigas como un grafo completamente conectado
entre los eventos.

En \cite{post2016third} se presenta una competencia de resolución de CTP con
instancias reales. En el trabajo se comparan los resultados obtenidos por varios
grupos competidores y en la página web se mantienen las instancias actualizadas
junto a los mejores resultados encontrados por el momento.

Por último, en \cite{rossi2002comparison} se hace una comparación del
rendimiento entre algoritmos básicos de las heurísticas descritas anteriormente
en la resolución del CTP. Los resultados indican que para instancias pequeñas
todos los algoritmos tienen un buen comportamiento, en especial ILS, SA y ACO.
Mientras las instancias incrementan su tamaño la que peor responde al cambio
es ACO mientras que para instancias muy grandes la mayoría de los algoritmos ni
siquiera logran cumplir la factibilidad. El algoritmo mejor evaluado en esta
comparación es ILS aunque no siempre llega a soluciones factibles. 

\section{Modelo}\label{sec:mod}
Digamos $E$ el conjunto de eventos a distribuir y $T$ el conjunto de espacios de
tiempo disponibles. Cada evento es una n-tupla de recursos, en este caso $E = P
\times C \times S$ con $p_{it} \in P$ una variable binaria que representa la
asignación del profesor $i$ en el tiempo $t \in T$, análogamente para las clases
en $c_{it} \in C$ y para las salas en $s_{it} \in S$.

Una solución será un conjunto de eventos $E$ de manera que se cumplan las
siguientes restricciones:
\begin{itemize}
  \item
    Cada recurso solo puede usarse una vez por tiempo, como los recursos son
    variables binarias si uno de ellos ya esta asignado no se puede usar
    nuevamente.
    $$ (p_{it}=1, p_{it}\in e_x \in E) \land (p_{ik}=1, p_{ik}\in e_y \in E,
    e_y \neq e_x) \iff t \neq k $$
    Análogo para las clases y salas.
  \item
    Un evento debe tener todos sus recursos asignados en un mismo tiempo.
    $$ \forall e \in E : e = (p_{pt}, c_{ci}, s_{sj}) \iff t = i = j $$
    por comodidad diremos que \it{assign}($e$) $= t$ cuando un evento $e$ tiene
    todos sus recursos asignados a un tiempo $t$.
  \item
    Todos los eventos deben estar asignados correctamente. Como la primera
    restricción impide la una asignación errónea nos basta verificar que todos
    los eventos estén asignados.
    $$ \forall e \in E : \it{assign}(e) = t \in T$$
\end{itemize}

Para trabajar con un algoritmo de hormigas es necesario generar un ``camino''
por el cual éstas puedan transitar, en esta ocasión utilizaremos un grafo
completo en el cual cada nodo es un evento y todo par de eventos está conectado.
Esta representación también es utilizada en \cite{socha2002max}.

En este tipo de algoritmo existen dos puntos fundamentales que determinan su
comportamiento. El primero es la probabilidad que tienen cada hormiga para ir de
un nodo a otro y el segundo es la función de actualización de las feromonas que
determinan esta probabilidad.

La función de probabilidad de típica (\cite{colorni1991distributed}) de ACO es
definida como sigue:
\begin{equation} \label{eq:prob}
  P_{ij}(t) = \frac{ [\tau_{ij}(t)]^{\alpha}\cdot [\eta_{ij}]^{\beta} }
  { \sum_{j=1}^{n} [\tau_{ij}(t)]^{\alpha}\cdot [\eta_{ij}]^{\beta} }
\end{equation}
Donde $P_{ij}$ es la probabilidad de ir del nodo $i$ al nodo $j$, en nuestro
caso, del evento $i$ al $j$. $\tau_{ij}$ es la feromona presente en dicho camino
y $\eta$ es una heurística local, en nuestro caso será la duración del evento
$j$ pues se espera asignar los eventos más largos primero. Los paramentros
$\alpha$ y $\beta$ modifican el comportamiento de la formula.

Por otro lado la función de actualización de las feromonas es la siguiente:
\begin{equation} \label{eq:pher}
  \tau_{ij}(t+z) = (1-\rho)\tau_{ij}(t) + \Delta \tau_{ij}
\end{equation}
Donde $0 \leq \rho \leq 1$ es el coeficiente de evaporación de la feromona y
$\Delta\tau$ será el cambio producido en ella cuando la hormiga transita el
camino.

Para modificar el comportamiento general del algoritmo se da la opción de
modificar los parámetros $\alpha, \beta, \Delta\tau$ y $\rho$ Además se espera
que exista una feromona inicial $\rho_0$ cercana a cero.

\section{Representación}\label{sec:repr}
Para representar el problema y sus soluciones se utilizan las estructuras y
datos descritos en esta sección.

Una \bf{instancia} contendrá todos los datos del problema, tiene un nombre y un
arreglo para cada uno de sus recursos además de los tamaños de estos. Cada
recurso tiene un nombre y un identificador. La instancia además contienen todos
los eventos que deben ser asignados, cada evento tiene el identificador de los
recursos que necesita, si puede usar cualquiera este campo se deja nulo.
La estructura de instancia se muestra en la figura~\ref{fig:inst} mientras que
la de eventos está en la figura~\ref{fig:event}

\begin{figure}[htpb]
  \centering
  \begin{tabular}{c}
    \lstinputlisting[language=c,firstline=8,lastline=14]{../src/structures.h}
  \end{tabular}
  \caption{Instancia.}
  \label{fig:inst}
\end{figure}

\begin{figure}[htpb]
  \centering
  \begin{tabular}{c}
    \lstinputlisting[language=c,firstline=17,lastline=21]{../src/structures.h}
  \end{tabular}
  \caption{Evento.}
  \label{fig:event}
\end{figure}

Cada hormiga puede generar una \bf{solución} por lo que estas deben ser
guardadas en una estructura individual. Una solución tendrá un puntero a la
instancia que soluciona, una tabla con los recursos asignados, el camino que
sigue la hormiga en el grafo, un arreglo con asignaciones de eventos y el valor
total de la solución. Cabe destacar que las tablas de recursos si bien podrían
ser booleanas se prefirió utilizar el indice del evento que utiliza el recurso
para hacer más fácil la detección de errores, el recurso se considera ocupado
si en la tabla tiene un valor distinto a \tt{NONE} (definido como -1).
Se puede ver la estructura solución en la figura~\ref{fig:result}.

\begin{figure}[htpb]
  \centering
  \begin{tabular}{c}
    \lstinputlisting[language=c,firstline=31,lastline=42]{../src/structures.h}
  \end{tabular}
  \caption{Solución.}
  \label{fig:result}
\end{figure}

Una asignación de un evento es un arreglo con la estructura descrita en la
figura~\ref{fig:assign}, el tamaño de este arreglo será la duración del evento
ya que algún evento puede ser completado en espacios de tiempo no consecutivos.
La asignación posee los identificadores de todos los recursos involucrados y un
valor local de asignación. El valor total de una solución es la suma de los
valores locales.

\begin{figure}[htpb]
  \centering
  \begin{tabular}{c}
    \lstinputlisting[language=c,firstline=25,lastline=28]{../src/structures.h}
  \end{tabular}
  \caption{Asignación.}
  \label{fig:assign}
\end{figure}

Digamos el grafo $G=(N,A)$, donde el conjunto $N$ representa los nodos, en
nuestro caso $E$ (los eventos); y el conjunto $A$ contiene los arcos, al ser
un grafo completo $(e_i, e_j) \forall i \neq j, e_x \in E$.

Además las feromonas son guardadas en un arreglo bi-dimensional de
$|E|\times|E|$ de manera que la feromona del arco $(e_i, e_j)$ esté en el dato
\tt{pheromone[i][j]}.

Por ejemplo, supongamos una instancia con 4 profesores, 4 clases y 4 salas a
asignar en 4 espacios de tiempos y un cierta cantidad de eventos. Los datos
quedarán representados en la figura~\ref{fig:ex:ins}.

\begin{figure}[h]
  \centering
  \begin{tabular}{c|lcl}
    \tt{instance} & \tt{name}       & \ra & \it{Ejemplo} \\
                  & \tt{times[]}    & \ra & 
          \{0: \it{Dia 1},\, 1: \it{Dia 2},\, 2: \it{Dia 3},\, 3: \it{Dia 4}\}\\
                  & \tt{teachers[]} & \ra & 
                      \{0: \it{T1},\, 1: \it{T2},\, 2: \it{T3},\, 3: \it{T4}\}\\
                  & \tt{rooms[]}    & \ra &
                      \{0: \it{R1},\, 1: \it{R2},\, 2: \it{R3},\, 3: \it{R4}\}\\
                  & \tt{classes[]}  & \ra &
                      \{0: \it{C1},\, 1: \it{C2},\, 2: \it{C3},\, 3: \it{C4}\}\\
                  & \tt{s\_time}    & \ra & 4 \\
                  & \tt{s\_teacher} & \ra & 4 \\
                  & \tt{s\_rooms}   & \ra & 4 \\
                  & \tt{s\_class}   & \ra & 4 \\
                  & \tt{event}      & \ra & 
                  \{0: \begin{tabular}{c|lcl}
                         \tt{event} & \tt{duration} & \ra & 1 \\
                                    & \tt{class}    & \ra & 0 \\
                                    & \tt{teacher}  & \ra & 1 \\
                                    & \tt{room}     & \ra & 3 \\
                       \end{tabular},
                    1: \dots \}\\
  \end{tabular}
  \caption{Ejemplo de instancia.}\label{fig:ex:ins}
\end{figure}

Los indices representan el identificador de cada recurso y evento, por lo tanto
el evento 0 ($e_0$) tiene duración 1 y necesita del profesor 1 (\it{T2}), la
sala 3 (\it{R4}) y el curso 0 (\it{C1}).

Si una solución elige y logra asignar recursos para el evento $e_0$ como el
primer evento de su camino en el tiempo 1 (\it{Dia 2}), sus matrices de
ocupación de recursos quedarán como se muestra en la figura~\ref{fig:ex:tab}.

\begin{figure}[htpb]
  \centering
  \begin{subfigure}[b]{.3\textwidth}
    \begin{equation*}
      \bordermatrix{
        ~          & \it{T1} & \it{T2} & \it{T3} & \it{T4} \cr
        \it{Dia 1} & -       & -       & -       & -       \cr
        \it{Dia 2} & -       & 0       & -       & -       \cr
        \it{Dia 3} & -       & -       & -       & -       \cr
        \it{Dia 4} & -       & -       & -       & -       \cr
      }
    \end{equation*}
    \caption{Profesor $\times$ Tiempo.}
    \label{fig:mt}
  \end{subfigure}
  \hspace{2mm}
  \begin{subfigure}[b]{.3\textwidth}
    \begin{equation*}
      \bordermatrix{
        ~          & \it{C1} & \it{C2} & \it{C3} & \it{C4} \cr
        \it{Dia 1} & -       & -       & -       & -       \cr
        \it{Dia 2} & 0       & -       & -       & -       \cr
        \it{Dia 3} & -       & -       & -       & -       \cr
        \it{Dia 4} & -       & -       & -       & -       \cr
      }
    \end{equation*}
    \caption{Sala $\times$ Tiempo.}
    \label{fig:mr}
  \end{subfigure}
  \hspace{2mm}
  \begin{subfigure}[b]{.3\textwidth}
    \begin{equation*}
      \bordermatrix{
        ~          & \it{R1} & \it{R2} & \it{R3} & \it{R4} \cr
        \it{Dia 1} & -       & -       & -       & -       \cr
        \it{Dia 2} & -       & -       & -       & 0       \cr
        \it{Dia 3} & -       & -       & -       & -       \cr
        \it{Dia 4} & -       & -       & -       & -       \cr
      }
    \end{equation*}
    \caption{Clase $\times$ Tiempo.}
    \label{fig:mc}
  \end{subfigure}
  \caption{Matrices de ocupación de recursos.}
  \label{fig:ex:tab}
\end{figure}

La asignación se registrará en el camino de la solución marcando como primer
nodo visitado $e_0$. Este proceso se repite hasta completar todos los eventos.

Si un evento no puede ser asignado por que sus recursos no tienen un tiempo
libre en común, se le asignará un valor de -1 a cada uno de sus tiempos no
asignados, así, si no se logra cumplir un recurso con tiempo 6 su valor total
será de -6 lo que indica que no se han cumplido 6 condiciones.


\section{Descripción del algoritmo}\label{sec:alg}
El algoritmo implementado consta de 3 pasos fundamentales para cada hormiga en
cada iteración.
\begin{enumerate}
  \item
    Elegir un nodo inicial aleatorio y asignarle recursos.
  \item
    Desde ese nodo elegir el nodo siguiente según la probabilidad dictada por la
    ecuación~\ref{eq:prob} y asignarle recursos.
  \item
    Si no se han visitado todos los eventos repetir el paso 2. En otro caso el
    valor de la solución es la suma del valor de las asignaciones.
\end{enumerate}

Cuando todas las hormigas han hecho sus caminos se compara el valor de sus
soluciones y solo aquellas con la mejor solución depositan feromonas según la
ecuación~\ref{eq:pher}.

Como podemos notar la asignación de recursos es un problema en sí. En esta
implementación la asignación se hace aleatoriamente entre los tiempos
candidatos de la siguiente manera:
\begin{enumerate}
  \item
    Se crea un set con todos los tiempos existentes.
  \item
    Por cada recurso requerido se verifica en la tabla los tiempos no
    disponibles y se eliminan del set.
  \item
    Si el tamaño del set resultante es mayor o igual la duración del evento se
    le asignan tiempos aleatorios entre los disponibles.
  \item
    Si el tamaño del set es menor se le da a la asignación un valor de -1 por
    cada tiempo no asignado (su duración).
\end{enumerate}

Este proceso está plasmado en el algoritmo~\ref{alg:assign}.

\begin{algorithm}
  \caption{Pseudo código para asignar recursos. \it{assign}($e_i$)}
  \label{alg:assign}
  \begin{algorithmic}[1]
    \Require Un evento $e_i$.
    \Ensure El evento con recursos asignados y un valor de asignación.
    \State \it{disponibles} \la \it{new\_set}() 
    \ForAll{tiempo \it{t}}
      \State \bf{append} \it{t} \ra \it{disponibles}
    \EndFor
    \ForAll{recurso \it{r} necesitado por $e_i$}
      \ForAll{tiempo \it{t}}
        \If{\it{r} tiene asignado el tiempo \it{t}}
          \State \bf{remove} \it{t} \bf{from} \it{disponibles}
        \EndIf
      \EndFor
    \EndFor
    \If{$|$\it{disponible}$|$ $\geq$ duración de $e_i$}
      \For{duración de $e_i$}
        \State \bf{pop} aleatorio a \it{disponibles} \ra t
        \ForAll{recurso \it{r} necesitado por $e_i$}
          \State Asignar el recurso $r$ al tiempo $t$ y actualizar su tabla.
        \EndFor
      \EndFor
      \State Valor de la asignación \la duración de $e_i$
    \Else
      \State Valor de la asignación \la -(duración de $e_i$)
    \EndIf
  \end{algorithmic}
\end{algorithm}

La optimización por camino de hormigas resultante queda reflejada en el
algoritmo~\ref{alg:ants}.

\begin{algorithm}
  \caption{Pseudo código del algoritmo de hormigas.}
  \label{alg:ants}
  \begin{algorithmic}[1]
    \Require Una instancia.
    \Ensure La mejor solución encontrada.
    \State \it{feromona}[\it{\#eventos}][\it{\#eventos}] \la
           \it{new\_array}(\it{init\_value} = $\tau_0$)
    \State \it{solución}[\it{\#hormigas}] = \it{new\_array}()
    \State \it{M} \la Mejor solución encontrada hasta el momento
    \For{\it{\#iteraciones}}
      \For{\it{i} \bf{from} 0 \bf{to} \it{\#hormigas}}
        \State \it{solución}[\it{i}] = \it{new\_result}()
        \State \it{actual} \la evento aleatorio
        \State \it{assign}(\it{actual})
        \While{Queden eventos sin visitar}
          \State \it{siguiente} \la elegir según la ecuación~\ref{eq:prob} entre
          los eventos aún no visitados.
          \State \it{assign}(\it{siguiente})
          \State agregar (\it{actual, siguiente}) al camino de esta solución.
          \State \it{actual} \la \it{siguiente}
        \EndWhile
      \EndFor
      \State \it{mejores} = \it{new\_array}()
      \For{\it{i} \bf{from} 0 \bf{to} \it{\#hormigas}}
        \If{\it{hormiga}[i].\it{valor} $\geq$ mejor valor de la iteración}
          \State \bf{append} \it{hormiga}[i].\it{camino} \ra \it{mejores}
        \EndIf
      \EndFor
      \State Aplicar evaporación según la ecuación~\ref{eq:pher}
      \ForAll{\it{camino} \bf{in} \it{mejores}}
        \State Agregar feromona al \it{camino}
      \EndFor
    \EndFor
    \State Imprimir \it{M}
  \end{algorithmic}
\end{algorithm}


\section{Experimentos}\label{sec:exp}
Para la experimentación se utilizaron las instancias más recientes de 
\cite{post2016third} disponibles en su pagina 
web\footnote{https://www.utwente.nl/ctit/hstt/archives/}.

Los archivos se dividen en instancias artificiales y reales, y se presentan en
formato XHSTT\cite{post2014xhstt} el cual da una gran libertad a la hora de
definir recursos y restricciones segundarías.

Debido a la complejidad que presentan algunas de estas instancias para este
trabajo se experimentó con un subconjunto de ellas que presentan las siguiente
características:
\begin{enumerate}
  \item
    La instancia tiene al menos los recursos \emph{time}, \emph{class} y
    \emph{teacher} y todos sus eventos tienen estos recursos asignados.
  \item
    Opcionalmente pueden tener el recurso \emph{room}.
  \item
    Los eventos pueden tener \emph{room} asignado o libre.
\end{enumerate}

Además solo se busca generar soluciones factibles por lo que las restricciones
planteadas por las instancias son ignoradas. Por ello además si se alcanza una
solución factible antes de completar las iteraciones el algoritmo terminará
inmediatamente.

Las instancias que cumplen con los requisitos planteados se presentan en la
tabla~\ref{tab:inst}. Los resultados de las mismas están disponibles en la
página web del
concurso\footnote{https://www.utwente.nl/ctit/hstt/archives/XHSTT-2014/overview.html}\footnote{https://www.utwente.nl/ctit/hstt/archives/XHSTT-2012/}.

\begin{table}[h]
\centering
\begin{tabular}{|l|r|r|r|r|r|r|}
  \hline
  \bf{Instancia} & \bf{Profesores} & \bf{Clases} & \bf{Salas} & \bf{Tiempos} &
  \bf{Eventos} & \bf{Duración Total}  \\\hline
  \bf{Sudoku4x4} &  4  &  4  &  4  &  4  &  16  &   16  \\\hline
  \bf{Brazil1}   &  8  &  3  &  0  & 25  &  21  &   75  \\\hline
  \bf{Brazil2}   & 14  &  6  &  0  & 25  &  63  &  150  \\\hline
  \bf{Brazil4}   & 23  & 12  &  0  & 25  & 127  &  300  \\\hline
  \bf{Brazil7}   & 33  & 20  &  0  & 25  & 205  &  500  \\\hline
  \bf{ORL-hdtt4} &  4  &  4  &  4  & 30  &  59  &  120  \\\hline
  \bf{ORL-hdtt6} &  6  &  6  &  6  & 30  & 125  &  180  \\\hline
  \bf{ORL-hdtt8} &  8  &  8  &  8  & 30  & 197  &  240  \\\hline
\end{tabular}
\caption{Instancias a solucionar.}\label{tab:inst}
\end{table}

Para todas estas instancias se probará con 5 y 20 hormigas para 5, 25, 100,
500, 1000 y 5000 iteraciones.

Además como parámetros se utilizará $\alpha = 1, \beta = 0.5, \tau_0 = 0.001,
\rho = 0.2$ y $\Delta\tau = 0.005$

\section{Resultados}\label{sec:res}
Los resultados con 5 hormigas se muestran en la tabla~\ref{tab:5ants} mientras
que los tiempos de ejecución están en la tabla~\ref{tab:5ants:time}.

\begin{table}[h]
\centering
\begin{tabular}{|l|r|r|r|r|r|r|}
  \hline
                 & 5   & 25  & 100 & 500 & 1000 & 5000  \\\hline
  \bf{Sudoku4x4} & 0   & -   & -   & -   & -    & -     \\\hline
  \bf{Brazil1}   & 0   & -   & -   & -   & -    & -     \\\hline
  \bf{Brazil2}   & -5  & -3  & 0   & -   & -    & -     \\\hline
  \bf{Brazil4}   & -21 & -20 & -14 & -11 & -10  & -13   \\\hline
  \bf{Brazil7}   & -28 & -24 & -24 & -19 & -24  & -22   \\\hline
  \bf{ORL-hdtt4} & -24 & -19 & -18 & -16 & -15  & -13   \\\hline
  \bf{ORL-hdtt6} & -26 & -27 & -27 & -27 & -25  & -25   \\\hline
  \bf{ORL-hdtt8} & -40 & -36 & -35 & -33 & -34  & -34   \\\hline
\end{tabular}
\caption{Resultados con 5 hormigas.}\label{tab:5ants}
\end{table}

\begin{table}[h]
\centering
\begin{tabular}{|l|r|r|r|r|r|r|}
  \hline
                 & 5         & 25 & 100   & 500    & 1000    & 5000    \\\hline
  \bf{Sudoku4x4} & 0,006 (3) & -  & -     & -      & -       & -       \\\hline
  \bf{Brazil1}   & 0,009 (2) & -  & -     & -      & -       & -       \\\hline
  \bf{Brazil2}   & 0,039 & 0,130 & 0,380 (77) & -  & -       & -       \\\hline
  \bf{Brazil4}   & 0,167 & 0,770 & 2,821  & 14,122 & 28,956  & 2:23,07 \\\hline
  \bf{Brazil7}   & 0,523 & 2,618 & 10,152 & 50,211 & 1:44,46 & 7:44,90 \\\hline
  \bf{ORL-hdtt4} & 0,035 & 0,121 & 0,462  & 2,264  & 4,444   & 22,688  \\\hline
  \bf{ORL-hdtt6} & 0,149 & 0,660 & 2,659  & 13,384 & 26,196  & 2:11,92 \\\hline
  \bf{ORL-hdtt8} & 0,446 & 2,158 & 8,879  & 43,332 & 1:28,74 & 7:17,94 \\\hline
\end{tabular}
\caption{Tiempos con 5 hormigas.}\label{tab:5ants:time}
\end{table}

Como podemos ver de estos resultados el algoritmo no tiene problemas para
encontrar soluciones a problemas pequeños aunque tenga que asignar también
salas, pero para los problemas más grandes no logra encontrar soluciones y aún
con mayor número de iteraciones los resultados no mejoran.
Por ello para los experimentos con 20 hormigas se cambio a $\Delta\tau = 0.008$
para mejorar la explotación.

\begin{table}[h]
\centering
\begin{tabular}{|l|r|r|r|r|r|r|}
  \hline
                 & 5   & 25  & 100 & 500 & 1000 & 5000  \\\hline
  \bf{Sudoku4x4} & 0   & -   & -   & -   & -    & -     \\\hline
  \bf{Brazil1}   & 0   & -   & -   & -   & -    & -     \\\hline
  \bf{Brazil2}   & -3  & 0   & -   & -   & -    & -     \\\hline
  \bf{Brazil4}   & -19 & -17 & -15 & -15 & -13  & -12   \\\hline
  \bf{Brazil7}   & -32 & -26 & -24 & -22 & -16  & -19   \\\hline
  \bf{ORL-hdtt4} & -18 & -16 & -15 & -16 & -13  & -14   \\\hline
  \bf{ORL-hdtt6} & -31 & -29 & -27 & -26 & -24  & -22   \\\hline
  \bf{ORL-hdtt8} & -39 & -37 & -33 & -34 & -33  & -32   \\\hline
\end{tabular}
\caption{Resultados con 20 hormigas.}\label{tab:20ants}
\end{table}

\begin{table}[h]
\centering
\begin{tabular}{|l|r|r|r|r|r|r|}
  \hline
                 & 5         & 25  & 100    & 500    & 1000    & 5000     \\\hline
  \bf{Sudoku4x4} & 0,006 (1) & -   & -      & -      & -       & -        \\\hline
  \bf{Brazil1}   & 0,008 (1) & -   & -      & -      & -       & -        \\\hline
  \bf{Brazil2}   & 0,109  & 0,256 (13) & -  & -      & -       & -        \\\hline
  \bf{Brazil4}   & 0,529  & 2,644  & 10,353 & 52,463 & 1:50,33 & 9:50,28  \\\hline
  \bf{Brazil7}   & 1,939  & 9,713  & 37,595 & 3:05,38& 6:13,34 & 34:29,47 \\\hline
  \bf{ORL-hdtt4} & 0,094  & 0,416  & 1,674  & 8,146  & 17,850  & 1:29,05  \\\hline
  \bf{ORL-hdtt6} & 0,525  & 2,495  & 9,905  & 51,038 & 1:39,73 & 8:54,07  \\\hline
  \bf{ORL-hdtt8} & 1,655  & 8,208  & 32,598 & 2:44,59& 5:23,99 & 26:36,66 \\\hline
\end{tabular}
\caption{Tiempos con 20 hormigas.}\label{tab:20ants:time}
\end{table}

Como podemos notar los resultados no varían notablemente. Era de esperar,
teniendo en cuenta trabajos como \cite{socha2003ant} y
\cite{rossi2002comparison}, que afirman el mal comportamiento de los algoritmos
basados en hormigas para instancias grandes.

Las únicas instancias que logran llegar a resultados factibles son aquellas en
las cuales los tiempos totales son bajos y se logra en muy pocas iteraciones,
este resultado es congruente con \cite{rossi2002comparison}.

\section{Conclusiones}\label{sec:conc}
Tal como concluyen trabajos previos (\cite{socha2003ant},
\cite{rossi2002comparison}) un algoritmo basado en hormigas con asignaciones
aleatorias deja mucho que desear a la hora de solucionar instancias grandes.

El comportamiento del algoritmo es bueno para instancias pequeñas, ya que, por
combinatoria, es muy posible caer en un resultado correcto; esto no es posible
para las instancias grandes. Al mejorar la explotación se logra que el resultado
mejore ligeramente, pero no hasta el punto de hacer la instancia factible.
Por otro lado, si se incrementa la exploración el algoritmo, éste solo buscará
aleatoriamente hasta encontrar buenos resultados. Este proceso es demasiado
costoso por lo que convendría encontrar la solución secuencialmente.

Los tiempos de ejecución dependen completamente del problema. Lo
determinante en este punto son la cantidad de eventos y su duración. Cuando un
evento dura más de uno, el algoritmo lo divide en una cantidad de sub-eventos
acorde a su duración, haciendo del tiempo total (la suma del tiempo de cada
evento) una medida clara de cuanto durará la ejecución. Además existe cierto
costo asociado a asignar más de un recurso (como en las instancias con salas
libres), pero no es tanto como se esperaba.

Dos factores pueden ser la base del mal desempeño del algoritmo: la elección de
parámetros y la asignación local. Con respecto a los parámetros no se logró
encontrar una asignación que destacara por sobre las demás a la hora de entregar
soluciones. Este es un fenómeno común y uno de los problemas de heurísticas con
muchos parámetros como las hormigas. Generalmente se requiere un largo proceso
de prueba y error para determinar los parámetros correctos.

La asignación local por su parte puede ser determinante en la generación de
soluciones y es un problema en si mismo. La ventaja de tener una asignación
aleatoria es la simpleza que presenta tanto en la implementación como en la
ejecución, pero por otro lado, aplicar una heurística en este proceso puede
mejorar notablemente los resultados.

Finalmente, se destaca la facilidad que tiene implementar un algoritmo de
hormigas cuando ya se ha generado un grafo por el cual ellas recorran. En
general, este proceso puede hacerse sin preocuparse demasiado por las
restricciones del problema si se tiene un buen modelo para verificarlas.
La complejidad radica en la generación misma del grafo, ya que no todo problema
puede ser visualizado de esta manera.

\bibliographystyle{ieeetr}
\bibliography{bibliography}

\end{document} 
% vim: set ts=2 sw=2 sts=2 tw=80 : %
